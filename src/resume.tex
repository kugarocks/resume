%!TEX TS-program = xelatex
%!TEX encoding = UTF-8 Unicode
% Awesome CV LaTeX Template for CV/Resume
%
% This template has been downloaded from:
% https://github.com/posquit0/Awesome-CV
%
% Author:
% Claud D. Park <posquit0.bj@gmail.com>
% http://www.posquit0.com
%
% Template license:
% CC BY-SA 4.0 (https://creativecommons.org/licenses/by-sa/4.0/)
%


%-------------------------------------------------------------------------------
% CONFIGURATIONS
%-------------------------------------------------------------------------------
% A4 paper size by default, use 'letterpaper' for US letter
\documentclass[11pt, a4paper]{awesome-cv}

% set font
\usepackage{fontspec}
\usepackage{xeCJK}
\usepackage{hyperref}
\usepackage[normalem]{ulem}
% \xeCJKsetup{PunctStyle=hangmobanjiao}
\setCJKmainfont{PingFang SC}
% \setCJKmainfont{Heiti SC}
% \setCJKmainfont{Source Han Serif SC}
% \punctstyle{hangmobanjiao}
% \setmainfont{PingFang SC}

% Configure page margins with geometry
\geometry{left=1.4cm, top=.8cm, right=1.4cm, bottom=1.8cm, footskip=.5cm}

% Color for highlights
% Awesome Colors: awesome-emerald, awesome-skyblue, awesome-red, awesome-pink, awesome-orange
%                 awesome-nephritis, awesome-concrete, awesome-darknight
\colorlet{awesome}{awesome-red}
% Uncomment if you would like to specify your own color
% \definecolor{awesome}{HTML}{3E6D9C}

% Colors for text
% Uncomment if you would like to specify your own color
% \definecolor{darktext}{HTML}{414141}
% \definecolor{text}{HTML}{333333}
% \definecolor{graytext}{HTML}{5D5D5D}
% \definecolor{lighttext}{HTML}{999999}
% \definecolor{sectiondivider}{HTML}{5D5D5D}

% Set false if you don't want to highlight section with awesome color
\setbool{acvSectionColorHighlight}{false}

% If you would like to change the social information separator from a pipe (|) to something else
\renewcommand{\acvHeaderSocialSep}{\quad\textbar\quad}

% set line spacing
\linespread{1.3}

% set header after name and university skip
\renewcommand{\acvHeaderAfterNameSkip}{1.5mm}
\renewcommand{\acvHeaderAfterUniversitySkip}{0mm}

% \hypersetup{
%     colorlinks=true,
%     linkcolor=blue,
%     filecolor=blue,      
%     urlcolor=blue,
%     pdfborderstyle={/S/U/W 1}
% }

%-------------------------------------------------------------------------------
%	PERSONAL INFORMATION
%	Comment any of the lines below if they are not required
%-------------------------------------------------------------------------------
% Available options: circle|rectangle,edge/noedge,left/right
% \photo[rectangle,edge,right]{./examples/profile}
\name{邓}{焕良}
% \university{华南农业大学}{软件工程}{2010-2014}
\position{后端开发工程师{\enskip\cdotp\enskip}Golang}
% \address{235, World Cup buk-ro, Mapo-gu, Seoul, 03936, Republic of Korea}

\mobile{15800038517}
\email{kugacola@gmail.com}
\dateofbirth{1992-02-09}
\homepage{www.kugarocks.com}
% \github{kugarocks}
% \linkedin{posquit0}
% \gitlab{gitlab-id}
% \stackoverflow{SO-id}{SO-name}
% \twitter{@twit}
% \skype{skype-id}
% \reddit{reddit-id}
% \medium{madium-id}
% \kaggle{kaggle-id}
% \hackerrank{hackerrank-id}
% \googlescholar{googlescholar-id}{name-to-display}
%% \firstname and \lastname will be used
% \googlescholar{googlescholar-id}{}
% \extrainfo{extra information}

% \quote{``Be the change that you want in the world."}


%-------------------------------------------------------------------------------
\begin{document}

% Print the header with above personal information
% Give optional argument to change alignment(C: center, L: left, R: right)
\makecvheader[C]

% Print the footer with 3 arguments(<left>, <center>, <right>)
% Leave any of these blank if they are not needed
\makecvfooter
  {\today}
  {}
  {\thepage}

%-------------------------------------------------------------------------------
%	CV/RESUME CONTENT
%	Each section is imported separately, open each file in turn to modify content
%-------------------------------------------------------------------------------
%-------------------------------------------------------------------------------
%	SECTION TITLE
%-------------------------------------------------------------------------------
\cvsection{关于我}

%-------------------------------------------------------------------------------
%	CONTENT
%-------------------------------------------------------------------------------
\begin{cvparagraph}

%---------------------------------------------------------
早期在 115 网盘使用 PHP 负责网盘业务的日常开发。
酷狗音乐时期,主要使用 Lua/PHP 开发业务接口和维护支付网关。
在欢聚时代,曾主导开发了活动模版,该项目至今已运行 5 年,上线了 400 多个活动。
在趣丸科技,担任的角色是 DevOps,负责海外 TTChat 业务的开发和维护,使用技术主要是 Go/K8S。

喜欢分享和开源,平时会写一些技术文章和小工具,
曾经提交过 \href{https://github.com/shadow-maint/shadow/pull/1086}{\uline{Pull Request}} 修复了一个 Linux 命令的 Bug。
是 Vim、Linux、macOS、TUI(如 \href{https://github.com/kugarocks/markdown-finder}{\uline{Markdown Finder}} 命令行工具) 的忠实粉丝。
对新技术和工具充满兴趣,例如这份简历是使用 Cursor + LaTeX 编写的。
\end{cvparagraph}

%-------------------------------------------------------------------------------
%	SECTION TITLE
%-------------------------------------------------------------------------------
\cvsection{教育经历}


%-------------------------------------------------------------------------------
%	CONTENT
%-------------------------------------------------------------------------------
\begin{cventries}

%---------------------------------------------------------
  \cventry
    {学士学位 - 本科} % Degree
    {华南农业大学 - 软件工程} % Institution
    {广州 - 五山} % Location
    {2010 - 2014} % Date(s)
    {
      \begin{cvitems} % Description(s) bullet points
        \item {大学英语六级,曾选修密码学,对加密算法有一定的了解。}
      \end{cvitems}
    }

%---------------------------------------------------------
\end{cventries}

%-------------------------------------------------------------------------------
%	SECTION TITLE
%-------------------------------------------------------------------------------
\cvsection{工作经历}


%-------------------------------------------------------------------------------
%	CONTENT
%-------------------------------------------------------------------------------
\begin{cventries}

%---------------------------------------------------------
  \cventry
    {高级开发工程师} % Job title
    {趣丸科技} % Organization
    {广州} % Location
    {2021.07 - 2023.11} % Date(s)
    {
      \begin{cvitems} % Description(s) of tasks/responsibilities
        \item
        {
          负责 TTChat 海外业务的开发和维护。
          项目采用的架构是微服务,大大小小的服务加起来有 100 多个,如用户管理、房间管理、心跳服务等。
          个人主导开发的服务主要有用户等级、付费通话。
          用到的技术主要是 Go/gRPC/K8S/Istio。
        }
        \item {
          早期提供 HTTP 接口使用的是 Gin 框架,后来改为 Envoy Proxy。
          好处是可以将 gRPC 服务暴露为 RESTful API,允许不同客户端以统一的方式访问服务。
          Envoy Proxy 还可以提供流量管理和监控追踪功能(Prometheus)。
        }
        \item {
          由于微服务的数量过多,每个服务都配有一个 Sidecar 容器,分配的资源往往过剩。
          所以后来对 100 多个微服务按业务划分进行了合并,减少 Sidecar 容器数量,节省机器资源。
        }
      \end{cvitems}
    }

%---------------------------------------------------------
  \cventry
    {高级开发工程师} % Job title
    {欢聚时代} % Organization
    {广州} % Location
    {2017.12 - 2021.05} % Date(s)
    {
      \begin{cvitems} % Description(s) of tasks/responsibilities
        \item
        {
          主导开发活动模板,从 0 到 1,可本地演示。
          业务发展过程中有⼤量的重复组件如抽奖、礼包领取、相关任务。
          活动模板直接把强业务相关的玩法通过配置后台提供通用组件,无需开发介⼊,即可完成⼀个完整的活动。
          项目⾄今已运行 5 年,接入了 400 多个活动,减少了⼤量开发成本。
        }
        \item
        {
          运营后台的公共组件开发。
          由于公司内部使用的 MySQL 和 Redis 与标准的 API 有差别,这部分需要通过扩展源码去让框架满足业务需求。
          这些组件都有完整的单元测试,通过私有仓库供各个业务部门使用。
          除此之外,对 RBAC 权限和 CAS 统⼀登录也进行了组件封装。
        }
        \item
        {
          PHP 服务迁移到 Java,⽤到的技术有 Spring Boot、Apollo、Kong。
          使⽤ Spring Boot 的 AOP 做单⼀机房的请求转发,从 Kong 中获取可⽤的服务列表,再进⾏请求转发。
        }
      \end{cvitems}
    }

%---------------------------------------------------------
  \cventry
    {后台工程师} % Job title
    {酷狗音乐} % Organization
    {广州} % Location
    {2015.09 - 2017.06} % Date(s)
    {
      \begin{cvitems} % Description(s) of tasks/responsibilities
        \item
        {
          支付⽹关的开发和维护。
          支付网关主要是对接第三⽅⽀付渠道,给下游业务方提供接口。
          项目是南北双机房部署的,Redis 采⽤的是原生集群,接⼝幂等且有失败重试的机制。
          为了防⽌第三⽅没有及时回调,会定时主动查询订单状态。
        }
        \item
        {
          酷币/⾳乐包/VIP相关业务。
          酷币和音乐包的接口是采用 Openresty 开发的,相当于⽀付⽹关的下游业务方。
          查询接口 QPS 单机房大概在 1K 左右,另外接口数据都是经过 AES 加密的。
        }
      \end{cvitems}
    }

%---------------------------------------------------------
  \cventry
    {研发工程师} % Job title
    {115 网盘} % Organization
    {东莞} % Location
    {2014.08 - 2015.07} % Date(s)
    {
      \begin{cvitems} % Description(s) of tasks/responsibilities
        \item
        {
          在网盘⽂件分享业务中,需要把成千上万的⽂件快照记录插⼊到⽤户⽂件快照表中。
          对于这种⼤量的 DB 插⼊需求,是通过分表和异步任务去完成的。
          这⾥⽤到的技术主要有 Gearman、Elasticsearch、Supervisor。
        }
        \item
        {
          开发短地址服务用于短信发送。
          短地址采⽤用的是 Base58 编码⽅方式,预先⽣生成⼀一批存在 SSDB 队列列中,然后打乱,
          短地址和⻓地址的映射直接以 KV 的形式存储在 SSDB 中。
        }
      \end{cvitems}
    }

%---------------------------------------------------------
\end{cventries}

%-------------------------------------------------------------------------------
%	SECTION TITLE
%-------------------------------------------------------------------------------
\cvsection{个人项目}


%-------------------------------------------------------------------------------
%	CONTENT
%-------------------------------------------------------------------------------
\begin{cventries}

%---------------------------------------------------------
  \cventry
    {\href{https://www.kugarocks.com}{\uline{https://www.kugarocks.com}}} % 项目地址
    {技术分享网站} % 标题
    {kugarocks} % 位置
    {2024 - 至今} % 日期
    {
      \begin{cvitems} % 描述
        \item {使用 \href{https://gohugo.io/}{\uline{Hugo}} 和 \href{https://thulite.io/}{\uline{Thulite}} 框架的静态网站,分享技术和开源项目,支持简体中文、繁体中文(香港)、英文。}
        \item {代码托管在 \href{https://github.com/kugarocks/kugarocks.github.io}{\uline{GitHub}},使用 GitHub Actions 和 Webhook 自动部署。}
      \end{cvitems}
    }

%---------------------------------------------------------
  \cventry
    {\href{https://www.kugarocks.com/zh-cn/linux/readme/}{\uline{https://www.kugarocks.com/zh-cn/linux/readme/}}} % 项目地址
    {Linux 系统知识} % 标题
    {kugarocks} % 位置
    {2024 - 至今} % 日期
    {
      \begin{cvitems} % 描述
        \item {书籍《Linux 命令行和 Shell 脚本编程大全》的总结和扩展。}
        \item {写文章时发现 useradd 命令会修改 /etc/default/useradd 文件的权限,参与 \href{https://github.com/shadow-maint/shadow/pull/1083}{\uline{Pull Request}} 测试,现已修复。}
        \item {测试期间发现 useradd 命令的配置内容有 bug,翻查了历史代码并提交了 \href{https://github.com/shadow-maint/shadow/pull/1086}{\uline{Pull Request}},现已合并。}
      \end{cvitems}
    }

%---------------------------------------------------------
  \cventry
    {\href{https://minichat.kugarocks.com}{\uline{https://minichat.kugarocks.com}}} % 项目地址
    {迷你聊天室} % 标题
    {kugarocks} % 位置
    {2024 - 至今} % 日期
    {
      \begin{cvitems} % 描述
        \item {使用 Cursor LLM 开发的迷你聊天室,服务端使用 Golang 的 WebSocket 实现。}
        \item {终端风格,无需登录,无需联系人,以最简单直接的方式在两台设备之间发送文本。}
        \item {聊天角色随机分配,支持小黄人、海贼王。}
      \end{cvitems}
    }

%---------------------------------------------------------
  \cventry
    {\href{https://www.minions.wiki}{\uline{https://www.minions.wiki}}} % 项目地址
    {小黄人百科} % 标题
    {GitHub Pages} % 位置
    {2024 - 至今} % 日期
    {
      \begin{cvitems} % 描述
        \item {使用 \href{https://gohugo.io/}{\uline{Hugo}} 和 \href{https://github.com/bul-ikana/hugo-cards}{\uline{hugo-cards}} 主题的静态网站,分享小黄人的百科信息。}
        \item {代码托管在 \href{https://github.com/kugarocks/minions.wiki}{\uline{GitHub}},使用 GitHub Actions 自动构建,挺好玩的。}
        \item {为 hugo-cards 主题开发了 Profile Sidebar 特性,并提交了 \href{https://github.com/bul-ikana/hugo-cards/pull/28}{\uline{Pull Request}},已合并。}
      \end{cvitems}
    }

%---------------------------------------------------------
\end{cventries}

%-------------------------------------------------------------------------------
%	SECTION TITLE
%-------------------------------------------------------------------------------
\cvsection{个人风格}


%-------------------------------------------------------------------------------
%	CONTENT
%-------------------------------------------------------------------------------
\begin{cventries}

%---------------------------------------------------------
  \cventry
    {} {} {} {}
    {
      \begin{cvitems}
        \item {对编程和技术的热爱十年如一。}
        \item {代码整洁,每一行代码长度不过超 80/120 个字符,函数不超过 100 行。}
        \item {会为核⼼代码编写单元测试,知道如何编写可测试的代码,结合 LLM 提高测试效率。}
        \item {良好的英语文档阅读能力,喜欢在 GitHub 上交流,\href{https://www.kugarocks.com/zh-cn/linux/appendix/term-1/}{\uline{积累专业词汇}}和\href{https://www.kugarocks.com/zh-cn/linux/say-my-name/}{\uline{寻找标准发音}}。}
        \item {喜欢各种提高效率或优雅的工具,如 Vim、LaTeX,虽然有学习本成,但能带来独一无二的快乐。}
        \item {作为项⽬负责⼈,能主动协调多个部门的工作,推动项⽬的进展,保证如期完成。}
      \end{cvitems}
    }

%---------------------------------------------------------
\end{cventries}

% %-------------------------------------------------------------------------------
%	SECTION TITLE
%-------------------------------------------------------------------------------
\cvsection{证书}

%-------------------------------------------------------------------------------
%	CONTENT
%-------------------------------------------------------------------------------
\begin{cvhonors}

%---------------------------------------------------------
  \cvhonor
    {大学英语六级 CET-6} % Name
    {} % Issuer
    {} % Credential ID
    {2013} % Date(s)

%---------------------------------------------------------
\end{cvhonors}



%-------------------------------------------------------------------------------
\end{document}
