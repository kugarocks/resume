%-------------------------------------------------------------------------------
%	SECTION TITLE
%-------------------------------------------------------------------------------
\cvsection{个人项目}


%-------------------------------------------------------------------------------
%	CONTENT
%-------------------------------------------------------------------------------
\begin{cventries}

%---------------------------------------------------------
  \cventry
    {\href{https://www.kugarocks.com}{\uline{https://www.kugarocks.com}}} % 项目地址
    {技术分享网站} % 标题
    {kugarocks} % 位置
    {2024 - 至今} % 日期
    {
      \begin{cvitems} % 描述
        \item {使用 \href{https://gohugo.io/}{\uline{Hugo}} 和 \href{https://thulite.io/}{\uline{Thulite}} 框架的静态网站,分享技术和开源项目,支持简体中文、繁体中文(香港)、英文。}
        \item {代码托管在 \href{https://github.com/kugarocks/kugarocks.github.io}{\uline{GitHub}},使用 GitHub Actions 和 Webhook 自动部署。}
      \end{cvitems}
    }

%---------------------------------------------------------
  \cventry
    {\href{https://www.kugarocks.com/zh-cn/linux/readme/}{\uline{https://www.kugarocks.com/zh-cn/linux/readme/}}} % 项目地址
    {Linux 系统知识} % 标题
    {kugarocks} % 位置
    {2024 - 至今} % 日期
    {
      \begin{cvitems} % 描述
        \item {书籍《Linux 命令行和 Shell 脚本编程大全》的总结和扩展。}
        \item {写文章时发现 useradd 命令会修改 /etc/default/useradd 文件的权限,参与 \href{https://github.com/shadow-maint/shadow/pull/1083}{\uline{Pull Request}} 测试,现已修复。}
        \item {测试期间发现 useradd 命令的配置内容有 Bug,翻查了历史代码并提交了 \href{https://github.com/shadow-maint/shadow/pull/1086}{\uline{Pull Request}},现已合并。}
      \end{cvitems}
    }

%---------------------------------------------------------
  \cventry
    {\href{https://github.com/kugarocks/markdown-finder}{\uline{https://github.com/kugarocks/markdown-finder}}} % 项目地址
    {Markdown Finder} % 标题
    {kugarocks} % 位置
    {2024 - 至今} % 日期
    {
      \begin{cvitems} % 描述
        \item {命令行工具,用于快速定位并复制 Markdown 中的代码块,可用于记录常用命令。}
        \item {支持对 Markdown 文档进行拆分,成生 Table of Content。}
        \item {支持使用 GitHub 仓库管理 Markdown 数据,可共享/切换源。}
        \item {Forked from \href{https://github.com/maaslalani/nap}{\uline{maaslalani/nap}},命令行代码块管理工具。}
      \end{cvitems}
    }

%---------------------------------------------------------
  \cventry
    {\href{https://minichat.kugarocks.com}{\uline{https://minichat.kugarocks.com}}} % 项目地址
    {迷你聊天室} % 标题
    {kugarocks} % 位置
    {2024 - 至今} % 日期
    {
      \begin{cvitems} % 描述
        \item {使用 Cursor LLM 开发的迷你聊天室,服务端使用 Golang 的 WebSocket 实现。}
        \item {终端风格,无需登录,无需联系人,以最简单直接的方式在两台设备之间发送文本。}
        \item {聊天角色随机分配,支持小黄人、海贼王。}
      \end{cvitems}
    }

%---------------------------------------------------------
\end{cventries}
