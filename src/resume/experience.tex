%-------------------------------------------------------------------------------
%	SECTION TITLE
%-------------------------------------------------------------------------------
\cvsection{工作经历}


%-------------------------------------------------------------------------------
%	CONTENT
%-------------------------------------------------------------------------------
\begin{cventries}

%---------------------------------------------------------
  \cventry
    {高级开发工程师} % Job title
    {趣丸科技} % Organization
    {广州} % Location
    {2021.07 - 2023.11} % Date(s)
    {
      \begin{cvitems} % Description(s) of tasks/responsibilities
        \item
        {
          负责 TTChat 海外业务的开发和维护。
          项目采用的架构是微服务,大大小小的服务加起来有 100 多个,如用户管理、房间管理、心跳服务等。
          个人主导开发的服务主要有用户等级、付费通话。
          用到的技术主要是 Go/gRPC/K8S/Istio。
        }
        \item {
          早期提供 HTTP 接口使用的是 Gin 框架,后来改为 Envoy Proxy。
          好处是可以将 gRPC 服务暴露为 RESTful API,允许不同客户端以统一的方式访问服务。
          Envoy Proxy 还可以提供流量管理和监控追踪功能(Prometheus)。
        }
        \item {
          由于微服务的数量过多,每个服务都配有一个 Sidecar 容器,分配的资源往往过剩。
          所以后来对 100 多个微服务按业务划分进行了合并,减少 Sidecar 容器数量,节省机器资源。
        }
      \end{cvitems}
    }

%---------------------------------------------------------
  \cventry
    {高级开发工程师} % Job title
    {欢聚时代} % Organization
    {广州} % Location
    {2017.12 - 2021.05} % Date(s)
    {
      \begin{cvitems} % Description(s) of tasks/responsibilities
        \item
        {
          主导开发活动模板,从 0 到 1,可本地演示。
          业务发展过程中有⼤量的重复组件如抽奖、礼包领取、相关任务。
          活动模板直接把强业务相关的玩法通过配置后台提供通用组件,无需开发介⼊,即可完成⼀个完整的活动。
          项目⾄今已运行 5 年,接入了 400 多个活动,减少了⼤量开发成本。
        }
        \item
        {
          运营后台的公共组件开发。
          由于公司内部使用的 MySQL 和 Redis 与标准的 API 有差别,这部分需要通过扩展源码去让框架满足业务需求。
          这些组件都有完整的单元测试,通过私有仓库供各个业务部门使用。
          除此之外,对 RBAC 权限和 CAS 统⼀登录也进行了组件封装。
        }
        \item
        {
          PHP 服务迁移到 Java,⽤到的技术有 Spring Boot、Apollo、Kong。
          使⽤ Spring Boot 的 AOP 做单⼀机房的请求转发,从 Kong 中获取可⽤的服务列表,再进⾏请求转发。
        }
      \end{cvitems}
    }

%---------------------------------------------------------
  \cventry
    {后台工程师} % Job title
    {酷狗音乐} % Organization
    {广州} % Location
    {2015.09 - 2017.06} % Date(s)
    {
      \begin{cvitems} % Description(s) of tasks/responsibilities
        \item
        {
          支付⽹关的开发和维护。
          支付网关主要是对接第三⽅⽀付渠道,给下游业务方提供接口。
          项目是南北双机房部署的,Redis 采⽤的是原生集群,接⼝幂等且有失败重试的机制。
          为了防⽌第三⽅没有及时回调,会定时主动查询订单状态。
        }
        \item
        {
          酷币/⾳乐包/VIP相关业务。
          酷币和音乐包的接口是采用 Openresty 开发的,相当于⽀付⽹关的下游业务方。
          查询接口 QPS 单机房大概在 1K 左右,另外接口数据都是经过 AES 加密的。
        }
      \end{cvitems}
    }

%---------------------------------------------------------
  \cventry
    {研发工程师} % Job title
    {115 网盘} % Organization
    {东莞} % Location
    {2014.08 - 2015.07} % Date(s)
    {
      \begin{cvitems} % Description(s) of tasks/responsibilities
        \item
        {
          在网盘⽂件分享业务中,需要把成千上万的⽂件快照记录插⼊到⽤户⽂件快照表中。
          对于这种⼤量的 DB 插⼊需求,是通过分表和异步任务去完成的。
          这⾥⽤到的技术主要有 Gearman、Elasticsearch、Supervisor。
        }
        \item
        {
          开发短地址服务用于短信发送。
          短地址采⽤用的是 Base58 编码⽅方式,预先⽣生成⼀一批存在 SSDB 队列列中,然后打乱,
          短地址和⻓地址的映射直接以 KV 的形式存储在 SSDB 中。
        }
      \end{cvitems}
    }

%---------------------------------------------------------
\end{cventries}
